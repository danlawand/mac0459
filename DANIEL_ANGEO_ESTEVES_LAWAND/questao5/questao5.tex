\documentclass[
	10pt,
	parskip=half-,		
	paper=a4,			
	english, portuguese		
	]{scrartcl}			
	
\textheight = 220mm		
\footskip = 2.2cm		
\usepackage[portuguese]{babel}
% \usepackage[cm-default]{fontspec}

\usepackage{xunicode}
\usepackage{xltxtra}



\usepackage{microtype}

\usepackage{amsmath}
\usepackage{amssymb}
\usepackage{amsthm}

\usepackage{listings}	

\usepackage[cmyk]{xcolor}	
 \definecolor{middlegray}{rgb}{0.5,0.5,0.5}
 \definecolor{lightgray}{rgb}{0.8,0.8,0.8}
 \definecolor{orange}{rgb}{0.8,0.3,0.3}
 \definecolor{yac}{rgb}{0.6,0.6,0.1}



% Syntaxhighlighting
\lstset{
   basicstyle=\scriptsize\ttfamily,
   keywordstyle=\bfseries\ttfamily\color{orange},
   stringstyle=\color{green}\ttfamily,
   commentstyle=\color{middlegray}\ttfamily,
   emph={square}, 
   emphstyle=\color{blue}\texttt,
   emph={[2]root,base},
   emphstyle={[2]\color{yac}\texttt},
   showstringspaces=false,
   flexiblecolumns=false,
   tabsize=2,
   numbers=left,
   numberstyle=\tiny,
   numberblanklines=false,
   stepnumber=1,
   numbersep=12pt,
   xleftmargin=15pt
}

\usepackage{paralist}

\title{Prova MAC0459 - 2021}
\author{\textbf{Professor}: Roberto Hirata\\%
        \textbf{Aluno}: Daniel Angelo Esteves Lawand \textbf{NUSP}: 10297693}

\date{14.\,01.\,2022}

% --------------------------------------

\begin{document}
\maketitle			
\section{Questão 5}
    \subsection*{1}
    Não assisti a todas as aulas. O motivo principal foram os problemas familiares que tive, excepcionalmente, nesse semestre.
    \subsection*{2}
    Sinto que consegui entender os conceitos em geral, mas não consegui me aprofundar muito. O fato das aulas serem remotas acabou servindo positivamente para que eu estivesse "presente" em algumas aulas que eu não conseguiria se fosse no presencial, porém, acredito que o ambiente de sala de aula presencial contribuiria mais ao meu aprendizado.
    \subsection*{3}
    Participei de todas as atividades em grupo. Acredito que fui muito participativo, apresentava minhas ideias e sugestões, fazia perguntas e tirava dúvidas. Acredito que essas atividades foram as que mais contribuiram ao meu aprendizado. O discord foi uma ótima ferramenta, serviu ao seu propósito.
    \subsection*{4}
    Não busquei conhecer mais profundamente algum método exposto em aula. 
    \subsection*{5}
    Em algumas aulas estive motivado em outras não. Vejo que se tivesse mais atividades práticas eu estaria mais motivado, digo isso porque as atividades em grupo foram bem legais e me senti bem motivado.
    \subsection*{6}
    A disciplina satisfez minhas expectativas, pois meu objetivo ao me matricular na disciplina era o de entender melhor o que é ciência de dados, e vejo que isso foi cumprido especialmente nos primeiros meses de aula. Entendo que a disciplina abordou questões um pouco mais amplas do que apenas "o que é ciência de dados", mas para mim cumpriu o essencial.
    \subsection*{7}
    Não vejo outro tópico que poderia ser abordado. Entendo que a disciplina foi uma forma de abrir os olhos às possibilidades que existem dentro da área de dados, nos fazendo conhecer alguns métodos.
    \subsection*{8}
    Creio que deveriam existir mais tarefas práticas ao longo do semestre. Vejo que elas não foram pesadas, mas poderiam ter sido melhor distribuídas ao longo do semestre.  Gostei da primeira tarefa, de elaborar as perguntas para cada dataset, foi uma forma muito interessante de introduzir-nos ao assunto. A segunda tarefa já foi mais elaborada, foi muito boa também e me diverti fazendo, foi muito bom ter pessoas de diferentes áreas e de diferentes níveis de domínio da área, pois todos puderam crescer um pouco. Por fim, vejo que poderia ter uma outra tarefa que se aprofundasse um pouco mais do que foi feito na tarefa dois. Entendo que quanto mais o aluno tiver que pôr a mão na massa, melhor ele vai estar aprendendo, mesmo que erre. Por isso entendo que poderiam ter mais tarefas práticas durante o semestre, mas que poderiam ser menores do que as que nós fizemos, "micro-tarefas" pois assim o aluno está sempre em contato com o conteúdo e não fica sobrecarregado com o nível de exigência da tarefa.
    
    \subsection*{9}
    Considerando as palestras do Dr. Po-Shen Loh e da MSc Mariane Gonzalez, acredito que foram bem compatíveis com o contexto geral da disciplina e foram bem motivadoras para o que aprendemos. 
    
    A palestra do Dr. Po-Shen Loh foi muito interessante e bem atual, pois tratou de como podemos, com tecnologia e modelos matemáticos, interpretar como o vírus está se espalhando pela sociedade.
    
    De maneira diferente, a palestra da MSc Mariane Gonzalez aborda como é a plataforma de dados no Banco PAN, ou seja, como os dados estão armazenados e arquitetados, como são consumidos e, por fim, como esses dados são visualizados.
    
    \subsection*{10}
    Acredito que meu desempenho foi regular na disciplina, tive alguns problemas pessoais (doença e falecimento de pessoas próximas) que interferiram na experiência em todas as disciplinas do meu semestre, e isso impactou diretamente na minha presença e participação nas aulas de MAC0459, contudo estive presente e participativo em todas as tarefas de grupo.
    
    Vejo que a disciplina correspondeu à minha expectativa, que era ter um contato mais próximo com ciência de dados, e me senti animado em continuar na área e explorar novos conhecimentos da área. Entrei na disciplina sabendo pouco de ciência de dados (e muito de algoritmos), e saio com conhecimento suficiente para ter autonomia em explorar novos conhecimentos fora da sala de aula.
    
    Portanto, vejo que por ter sido um semestre desafiador e eu ter resistido, por eu ter cumprido o que me foi exigido e sentido que a disciplina cumpriu seu propósito, eu me daria a nota 7, pois fiz mais que o mínimo para passar, porém tive dificuldades que me impediram de me envolver de maneira mais profunda na matéria. 

\end{document}